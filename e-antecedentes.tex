\subsection*{Detector de emociones utilizando OpenCV}
El usuario de medium.com, Karan  Sethi, publicó un artículo donde presenta el procedimiento y resultados de la deteccion de sus rostro y el reconocimiento de 5 emociones. El usuario utilizó el software de OpenCV para la visión por computadora y Keras para el aprendizaje automático. El procedimiento que utilizó se reduce en dos grandes etapas, la creación del modelo, que incluye el entrenamiento del mismo, y la implementación del modelo para el reconocimiento de emociones en tiempo real \cite{Karan}.

En el artículo, Karan, dice explicitamente que los prerrequisitos para poder realizar este proyecto es tener conocimiento básico de los sigueintes temas:
\begin{itemize}
\item Python
\item OpenCV
\item Red neuronal de convolución (CNN)
\item numpy
\end{itemize}

El objetivo principal de Karan es crear una red neuronal de convolución utilizando Keras, la API para Python sobre (\textit{Deep Learning}), para detectar emociones en tiempo real a través de la realimentación al sistema de una cámara en tiempo real \cite{Karan}.

\subsection*{Sistema de detección de rostro y reconocimiento de gestos para robot animatrónico}
Anteriormente en la Universidad del Valle de Guatemala (UVG), el estudiante Luis Eduardo Ruano Argueta trabajó un software para la detección del rostro y emociones. Para este proyecto se utilizó Python, OpenCV, Base de datos de rostros, entre otros. Se utilizan dos algoritmos distintos para clasificar las distintas emociones y ellos utilizan una cámara Logitech 920 para obtener las imágenes en tiempo real \cite{Ruano2019Tesis}.

El objetivo principal de este proyecto es implementar por medio de software un programa que pueda darle seguimiento a los rostros y el reconocimiento de expresiones faciales de las personas. Entre los resultados se reportan distancias máximas para la detección del rostro, detección del rostro en ambiente oscuro y resultados para ambientes controlados y no controlados \cite{Ruano2019Tesis}.