La animatrónica es el proceso de fabricación y programación de mecanismos robóticos que son capaces de replicar el movimiento y comportamiento de seres vivos. En este proyecto se replica el movimiento de una cabeza humana y el comportamiento antes las emociones de otro ser humano. Anteriormente se había trabajado el modelo 3D del rostro animatrónico, con un alto parecido a la anatomía humana y este es capaz de realizar movimientos bastante allegados a la realidad de un rostro. 

Actualmente se trabaja en el comportamiento del mismo, con el objetivo de que pueda detectar el rostro de otra persona y reconocer las emociones que este presenta. Por ello se necesita crear un software que permita capturar imágenes en tiempo real y analizarlas de la misma manera para obtener información relevante que luego será procesada, por una red neuronal, para dar una respuesta estimulante para el usuario. Además de sincronizar los movimientos del rostro con la respuesta que se está dando para así poder replicar el comportamiento deseado.

El detector de emociones consta de dos etapas, la creación del modelo y la implementación del mismo. Se busca crear un modelo que reconozca las siete emociones básicas del ser humano y que está sea la raíz para poder desarrollar las respuestas del robot. Estas respuestas se procesarán en una red neuronal que puede modificarse o ampliarse en el futuro, para que este proyecto se pueda continuar trabajando y pueda aprovecharse, no solo para la Universidad del Valle de Guatemala, sino también para otras instituciones.