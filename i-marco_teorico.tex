\subsection*{Las 7 emociones básicas}
Las emociones evolucionaron para cumplir la función de promover la supervivencia física, pero el desarrollo de la cultura humana también ha impulsado la evolución de las emociones en el sentido de que estas cumplan funciones relativas a metas sociales, tales como llevarse bien con los demás y avanzar en la comunidad. Las emociones básicas tienen funciones sociales además de las de supervivencia \cite{rulicki2012cnv}.

Estas siete emociones básicas son:
\begin{itemize}
\item \textbf{Alegría:} Sensación dichosa de placer y bienestar \cite{rulicki2012cnv}.
\item \textbf{Tristeza:} Sensación opresiva de pérdida o carencia que produce desánimo \cite{rulicki2012cnv}.
\item \textbf{Miedo:} Sensación de agitación causada por una percepción de peligro debida a riesgos físicos, morales, o la presencia de dolor \cite{rulicki2012cnv}.
\item \textbf{Enojo:} Sensación perturbadora que resulta de una ofensa, una torpeza propia o un obstáculo natural. Generalmente incluye el deseo de reaccionar agresivamente \cite{rulicki2012cnv}.
\item \textbf{Asco:} Sensación de repugnancia debida a la percepción de un estímulo desagradable a los sentidos \cite{rulicki2012cnv}.
\item \textbf{Desprecio:} Sensación de rechazo o desestimación hacia otra persona o cosa, por considerarla inferior, indigna o carente de valor \cite{rulicki2012cnv}.
\item \textbf{Sorpresa:} Sensación súbita e inesperada de asombro \cite{rulicki2012cnv}.
\end{itemize}

\subsubsection*{Las señales faciales de las emociones}
Las señales faciales son las configuraciones de los distintos rasgos particulares de cada emoción, que son producidas por movimientos involuntarios en los músculos del rostro. Según Ekman, las expresiones de las emociones nos dan información acerca de lo que está ocurriendo dentro de la persona, lo que posiblemente ocurrió antes de que se desarrollara la emoción y aquello que puede llegar a pasar en consecuencia de la emoción. Para esto se utiliza el sistema FACS (Facial Action Coding System), que es un sistema de codificación que recoge todos movimientos expresivos del rostro en unidades de acción (AU) \cite{ekman2017rostro}.

\begin{table}[H]
\centering
\begin{tabular}{|c|c|}
\hline
\textbf{AU} & \textbf{Acción} \\ \hline
1             & Levantamiento interior de ceja      \\ \hline
2             & Levantamiento exterior de ceja      \\ \hline
4             & Bajar cejas                         \\ \hline
5             & Levantamiento del párpado superior  \\ \hline
6             & Levantamiento de mejillas           \\ \hline
7             & Apretar párpados                    \\ \hline
8             & Labios encimados uno de otro        \\ \hline
9             & Arrugar nariz                       \\ \hline
10            & Levantamiento del labio superior    \\ \hline
11            & Profundidad nasolabial              \\ \hline
12            & Tiramiento labial esquinal          \\ \hline
13            & Tiramiento labial frontal           \\ \hline
14            & Hoyuelo facial                      \\ \hline
15            & Depresión labial esquinal           \\ \hline
16            & Depresión labial frontal            \\ \hline
17            & Levantamiento de barbilla           \\ \hline
18            & Arruga labial                       \\ \hline
19            & Muestro de lengua                   \\ \hline
20            & Apretar los labios                  \\ \hline
21            & Apretamiento de cuello              \\ \hline
22            & Embudo labial                       \\ \hline
23            & Morder labios                       \\ \hline
24            & Presión labial                      \\ \hline
25            & Deslizamiento labial                \\ \hline
26            & Caída de la mandíbula               \\ \hline
27            & Apretamiento bucal                  \\ \hline
28            & Lamido labial                       \\ \hline
29            & Tracción de la mandíbula            \\ \hline
30            & Deslizamiento de mandíbula          \\ \hline
31            & Contracción mandibular              \\ \hline
32            & Mordida labial                      \\ \hline
33            & Succión de mejillas                 \\ \hline
34            & Inflar mejillas                     \\ \hline
35            & soplido de mejillas                 \\ \hline
36            & Protuberancia de lengua             \\ \hline
37            & Limpieza labial                     \\ \hline
38            & Dilatado nasal                      \\ \hline
39            & Compresión nasal                    \\ \hline
\end{tabular}
\caption{Lista de Unidades de acción y descripciones de acción.}
    \label{cuadro:AU}
\end{table}

\subsection*{Como reconocer las emociones en los demás}

\subsubsection*{Alegría}
El truco para reconocer correctamente la alegría está en los ojos. Concretamente, en las temidas patas de gallo. Si no aparecen estas características arrugas en el contorno exterior de los ojos, la sonrisa no se considera espontánea, sino una sonrisa social o intencionada. Los dos movimientos musculares o unidades de acción más características de la alegría son la elevación simétrica de las comisuras de los labios (AU12) y el ascenso de las mejillas (AU6).

\subsubsection*{Sorpresa}
Las tres unidades de acción más características de la sorpresa son la elevación simétrica de las cejas hacia el exterior (AU2), la apertura desorbitada de los párpados (AU5) y la caída de la mandíbula (AU26). El truco para reconocer correctamente la sorpresa está en los ojos y en la mandíbula. Los ojos parecen desorbitar, por efecto de la subida de los párpados superiores. Y lo que es más curioso, queda al descubierto la parte blanca de la esclerótica por encima del iris, que normalmente no vemos.

\subsubsection*{Tristeza}
Las unidades de acción más características de la tristeza son la elevación de las cejas hacia el interior (AU1), la caída de las comisuras de los labios (AU15), y la subida del mentón (AU17). El truco para reconocer correctamente la tristeza está en el comportamiento de las cejas. Lo más habitual es que, al subir hacia el interior, la activación del músculo frontal forme unas arrugas horizontales en el centro de la frente.

\subsubsection*{Miedo}
El truco para reconocer correctamente el miedo está en fijarnos bien en los ojos, y no confundirlos con los de la sorpresa. Las dos unidades de acción más características del miedo son la elevación de los párpados superiores (AU5), y la retracción o estiramiento horizontal de los labios (AU20).

\subsubsection*{Ira}
Las tres unidades de acción más características de la ira son juntar y bajar las cejas sobre la nariz (AU4), la tensión en los párpados inferiores (AU7) y la proyección de la mandíbula hacia adelante (AU29). El truco para reconocer correctamente la ira está en el ceño fruncido, formado por las típicas arrugas sobre la nariz al juntar y bajar las cejas. 

\subsubsection*{Asco}
Las dos unidades de acción más características del asco son arrugar la nariz (AU9), y elevar el labio superior (AU10). El truco para reconocer correctamente el asco está en la activación del pliegue nasolabial, fácilmente reconocible porque el labio superior asciende y la nariz se arruga.

\subsubsection*{Desprecio}
La única unidad de acción características del desprecio es la retracción de una de las comisuras de los labios hacia la mejilla (AU 14), formando el típico hoyuelo en un solo lado de la cara, o acentuando su existencia.El truco para reconocer correctamente el desprecio está en el típico hoyuelo, formado en una sola de las mejillas cuando los labios se retraen hacia un lado de la cara. El problema está en que no siempre se aprecia con claridad, sobre todo si la microexpresión es leve y rápida.
