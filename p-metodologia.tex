Lo primero en realizar serán las modificaciones y arreglos al modelo mecánico del rostro animatrónico. De esta manera se podrá probar el funcionamiento del mismo para saber que movimientos son posibles de realizar y así tener una idea de como tendrá que comportarse. Con las piezas diseñadas e impresas se procederá a desarmar el modelo y reemplazar todas las piezas defectuosas. Con el Rostro reconstruido se pondrá a funcionar para verificar que todo esté funcionando correctamente.

Seguido, se trabajará en el software para el reconocimiento de imágenes, especificamente en el reconocimiento de rostros humanos. Para ello se utilizará un software de visión por computadora y se realizarán pruebas para que este pueda detectar y seguir el rostro de distintas personas. Luego, se empezará con el reconocimiento de emociones, apra esto se utilizará un software que nos permita realizar aprendizaje profundo y de esta manera poder entrenar el modelo utilizando galerias disponibles de imágenes con emociones faciales. 

Por último, cuando el software esté funcionando correctamente, se sincronizarán los movimientos del rostro con las respuestas a las emociones. Es decir, el animatrónico moverá sus ojos o cuello para poder ver directamente al rostro del usuario, además de poder mover los labios para reaccionar a las emociones del usuario y poder mover el resto de rasgos faciales para que su interacción sea lo más apegado a una interacción humana. 

Adquirir una cámara para detectar las imágenes.

Implementar cámara al rostro para que se mueva con el.