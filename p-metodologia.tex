Lo primero en realizar serán las modificaciones y arreglos al modelo mecánico del rostro animatrónico. Estas modificaciones incluyen la base del rostro animatrónico, por una que sea más resistente y más eficiente que la que se tiene actualmente. De esta manera se podrá probar el funcionamiento del mismo para saber que movimientos son posibles de realizar y así tener una idea de como tendrá que comportarse. Con las piezas diseñadas e impresas se procederá a desarmar el modelo y reemplazar todas las piezas defectuosas, esto incluye modelos 3D, tornillos, tuercas y todas las piezas mecánicas necesarias para la reconstrucción del rostro. Con el rostro reconstruido se pondrá a funcionar los servomotores para verificar que todo esté funcionando correctamente, que movimientos son capaces de realizar y si todos los componentes electrónicos están funcionando o si hay alguno que deba cambiarse.

Seguido, se trabajará en el software para el reconocimiento de imágenes, especificamente en el reconocimiento de rostros humanos. Para ello se utilizará un software de visión por computadora y se realizarán pruebas para que este pueda detectar y seguir el rostro de distintas personas.El hardware que se estará utilizando en estas pruebas será la cámara de un teléfono inteligente mediante el software de DroidCam para obtener la imagen en la computadora. Luego, se empezará con el reconocimiento de emociones, para esto se utilizará un software que nos permita realizar aprendizaje profundo y de esta manera poder entrenar el modelo utilizando galerias disponibles de imágenes con emociones faciales. Para verificar el funcionamiento del reoconocimiento de emociones, se pondrá a prueba con distintas personas mostrando distintas emociones, esto para verificar que funciona con los distintos rostros.

Por último, cuando el software esté funcionando correctamente, se sincronizarán los movimientos del rostro con las respuestas a las emociones. Es decir, el animatrónico moverá sus ojos o cuello para poder ver directamente al rostro del usuario, además de poder mover los labios para reaccionar a las emociones del usuario y poder mover el resto de rasgos faciales para que su interacción sea lo más apegado a una interacción humana. Cabe recordar que el rostro tiene que ser capaz de mostrar emociones por medio de sus rasgos faciales. Por lo que es importante que este pueda reaccionar con una emocion distinta según sea el caso que se le presente al momento de interactuar con los usuarios. También se tienen que sincronizar los movimientos de los labios respecto a las silabas de la respuesta del rostro.

Para el resultado final del rostro se deberá conseguir una cámara que pueda ser capaz de realizar las mismas acciones que en las pruebas, pero que se pueda montar en el rostro, como por ejemplo una cámara web. Una cámara web es fácil de montar y pueden llegar a medir 6cm de alto, lo cual no está mal. Además que pueden llegar a tener una alta resolución con una alta tasa de frames por segundo.

Al momento que el rostro esté funcionando correctamente, es decir que funciona el reconocimiento de emociones y las respuestas a la misma, se pocederá a incorporar la cámara al rostro. La cámara deberá ir montada en el rostro para asegurarnos de que siempre tome el ángulo correcto. Además esta deberá girar con respecto al cuello para buscar la manera de mantener el rostro del usuario siempre centrado a la cámara. El montaje puede ser directamente en el rostro o a la base del mismo siempre que esta se pueda mover.